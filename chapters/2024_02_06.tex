\chapter[The 6$^{\text{th}}$ of February 2024 - Method papers - Goals \& quick description]{Method papers - Goals \& quick description}
\label{chap:MethodPapers}
\begin{chapabstract}
    The chapter recaps the meeting where we outline the objectives of the two method papers we aim to submit before summer.
\end{chapabstract}

\minitoc

\section{Method papers}

The first objective of 2024 is to set up the required tools to speed up both the micro and macro problems. The HiDeNN framework \parencite{zhang_hierarchical_2021} is a suitable candidate to give reduced models at both scales.

Some developments are still required to have a suitable framework for our needs. These developments are expected to be promoted through two \emph{method papers} described in the following sections.

\section{PDE solving - Mixed formulation}
The objective of that first paper would be to present
\begin{itemize}
    \item Optimised training strategy 
    \item Use a Mixed-formulation (PINN form of loss function)
    \item ``Element-wise'' implementation
\end{itemize}

Some open questions / possible work could include working on 
\begin{itemize}
    \item Regularisation of the Loss / Preventing collision of the nodes (negative Jacobian) to occur
\end{itemize}

The closest milestones are
\begin{itemize}
    \item Quadratics shape functions in 1D
    \item Definitions of shape functions suitable for 2D (see \cref{chap:TD})
\end{itemize}

\section{Parametric PDE}

An essential aspect of the required tool is providing a reduced-order parametric field model. 
\subsection{Short term}
This paper will rely on the tensor decomposition (TD) presented by \cite{zhang_hidenn-td_2022} and described in \cref{chap:TD}. It will focus on 
\begin{itemize}
    \item Implementing different parameters in the TD (as opposed to only spatial parameters as in \parencite{zhang_hidenn-td_2022}).
    \item Adapt the number of modes of the tensor decomposition of the fly (\emph{a priori ROM})
    \item Automate the non-linear interpolation using non-linear combinations of the modes (as opposed to solely using linear combinations of the modes as done in PGD, for instance)
    \begin{itemize}
        \item Such an optimisation could be based on a fully connected small NN in between the modes and the solution layer.
        \item $\vect{u}\left(\textcolor{BleuLMS!70}{\vect{x}}, \textcolor{LGreenLMS}{\left\{\mu_i\right\}_{i \ in \llbracket 1, \beta \rrbracket}}\right) = \sum\limits_{i=1}^m \textcolor{BleuLMS!70}{f\left(\overline{\vect{u}}_i(\vect{x})\right)} ~\textcolor{LGreenLMS}{\prod_{j=1}^{\beta}g\left(\lambda_i^j(\mu^j)\right)}$ 
        With $f$ and $g$ arbitrary functions of the modes, learned by the neural network in the reduced-order space so that the learning cost is lower but we achieve better reductibility by using non-linear combinations of the modes instead of increasing drastically the number of modes required.
        \item Take up the general idea put forward by \cite{kramer_learning_2024,geelen_learning_2024} but extend it to automate the non-linear combination  without prior knowledge of the required interpolation as opposed to guessing polynomial combinations
    \end{itemize}
\end{itemize}

A nice bonus would include the aspects of parametric spatial dependency described in \cref{chap:TD} in \cref{sec:SpatialParameters}.

\subsection{Mid term}
\label{ParaRefinement}
If the parameters' space is a tensor product of 1D spaces, then a localised refinement (as illustrated in \cref{fig:LocalRefinement}) of the parametric space is impossible, which would render the parametric description too heavy (as illustrated in \cref{fig:TDRefinement}). A specific strategy to describe the parameter space is required for localised refinement. 
\begin{itemize}
    \item High dimension shape functions for the hypercube parameters for instance
    \item What would be the impact of a pre-processing fully connected layer in between the input of the parameters and the 1D shape functions of the parameters?
    \begin{itemize}
        \item The parameters would not be treated independently from one another anymore as the pre-processing layer would connect them. What impact would that have on the Tensor Decomposition?
    \end{itemize}
\end{itemize}

\begin{figure}
\begin{subfigure}[t]{0.5\linewidth}
    \centering
    	\begin{tikzpicture}[scale=1.5]
		
		 % Orthonormal frame of reference
		\draw[->] (0,0) -- (4.5,0) node[below right] {$\mu_1$};
		\draw[->] (0,0) -- (0,3.5) node[above left] {$\mu_2$};
		
		% Draw coarse mesh
		\foreach \x in {0,1,2,3}{
			\foreach \y in {0,1,2}{
				\draw[BleuLMS] (\x,\y) rectangle (\x+1,\y+1);
			}
		}
		
		% Refinement region
		\draw[GreenLMS, very thick] (1,1) rectangle (2,2);
		
		% Draw refined mesh
		\foreach \x in {1}{
			\foreach \y in {1}{
				\draw[GreenLMS, thick] (\x,\y) rectangle (\x+0.5,\y+0.5);
			}
		}
				\foreach \x in {1.5}{
			\foreach \y in {1.5}{
				\draw[GreenLMS, thick] (\x,\y) rectangle (\x+0.5,\y+0.5);
			}
		}
		
		% Draw rerefined mesh
		\foreach \x in {1}{
			\foreach \y in {1}{
				\draw[GreenLMS, thick] (\x,\y) rectangle (\x+0.25,\y+0.25);
			}
		}
		\foreach \x in {1.25}{
	\foreach \y in {1.25}{
		\draw[GreenLMS, thick] (\x,\y) rectangle (\x+0.25,\y+0.25);
	}
}

	\end{tikzpicture}
    \caption{Local refinement in a 2D parametric space}
    \label{fig:LocalRefinement}
\end{subfigure}
  \begin{subfigure}[t]{0.5\linewidth}
    \centering
    	\begin{tikzpicture}[scale=1.5]
		
		% Orthonormal frame of reference
		\draw[->] (0,0) -- (4.5,0) node[below right] {$\mu_1$};
		\draw[->] (0,0) -- (0,3.5) node[above left] {$\mu_2$};
		
		% Draw coarse mesh
		\foreach \x in {0,1,2,3}{
			\foreach \y in {0,1,2}{
				\draw[BleuLMS] (\x,\y) rectangle (\x+1,\y+1);
			}
		}
		% Refinement region
		\draw[GreenLMS, very thick] (1,1) rectangle (2,2);
		\foreach \y in {0,1,2}{
		\draw[BleuLMS] (1,\y) rectangle (1.25,\y+1);
		}
		\foreach \y in {0,1,2}{
			\draw[BleuLMS] (1.5,\y) rectangle (2,\y+1);
		}
		
		\foreach \x in {0,1,2,3}{
			\draw[BleuLMS] (\x,1) rectangle (\x+1,1.25);
		}
		\foreach \x in {0,1,2,3}{
			\draw[BleuLMS] (\x,1.5) rectangle (\x+1,2);
		}
	

	\end{tikzpicture}
    \caption{Constrained refinement in two 1D-spaces tensor product}
    \label{fig:TDRefinement}
\end{subfigure}  
\caption{Refinement possibility of a 2D parametric space compared to its counterpart decomposed as a tensor product of two 1D spaces}
\end{figure}

\Rq{Each PGD modes are computed on an independent mesh, hence the localised correction of latter modes can lead to specific mesh refinement. \newline \noindent \textcolor{LGreenLMS}{\faLightbulb} This could solve the issue mentioned above.}