\chapter[The 29$^{\text{th}}$ of March 2024 - 2D and parametrisation]{2D and parametrisation}


\begin{chapabstract}
	Technical and strategy discussions about the 2D mixed formulation and the 
\end{chapabstract}


\minitoc



\section{2D}

\subsection{Convergence study}

\begin{itemize}
	\item Investigate the convergence propriety of the method as well as its efficiency when refining the mesh.
	\item Leverages the transfer learning of the HiDeNN framework to gain efficiency while refining (see first results in \cref{chap:Transfer_learning}).
\end{itemize}

\subsection{Toy examples guidelines}

\begin{itemize}
	\item Maybe try to avoid singularities
	\begin{itemize}
		\item I think that they should be handled as well as in the FEM framework
		\item But should be careful about how to handle BCs for the stress interpolation network
	\end{itemize}
\end{itemize}


\section{Parametrisation of the diseased regions}

\begin{itemize}
	\item \emph{A priori} good idea to use the parametrised functions instead of a full tiling of the space \cref{sec:BoundaryParam}.
	\item Rely on the level set functions idea to parametrise the functions is a 2D case with a very coarse underlying mesh.
	\begin{itemize}
		\item MMC (Moving Morphable Components \parencite{guo_doing_2014}) based on level-set may be a suitable framework
	\end{itemize}
\end{itemize}

\section{Writing strategy for the papers}

\begin{itemize}
	\item Start writing the scale 0 of the papers 
	\begin{itemize}
		\item Title 
		\item Outline
		\item Figures
	\end{itemize}
\end{itemize}