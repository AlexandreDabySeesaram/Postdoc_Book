\chapter[The 29$^{\text{th}}$ of February 2024 - Parametric investigation]{Parametric investigation}
\begin{chapabstract}
    Recaps of the novelty and investigations: 2- and 3-parameter NN-PGD
\end{chapabstract}

\minitoc

\section{NeuROM - Parametric reduced-order model}
\subsection{1 extra parameter}
Varying the overall stiffness on the hole structure.
\begin{itemize}
    \item First implementation of a NN-PGD (not automatically Greedy yet) \cref{chap:TD} in \cref{1D_NeuROM}
    \begin{itemize}
        \item The error indicator provided by the mixed formulation might be useful for on the fly decision of addition of new modes or not
    \end{itemize}
    \item Might need to rethink the Module architecture if no way to parallelise using \code{vmap} (Vectorisation of Modules).
    \item Advantage of classical PGD, only the loss has to be re-written for a new problem
    \item Initialisation from coarser ROM (coarser mesh and/or fewer modes) in \cref{chap:Transfer_learning}.
    \begin{itemize}
        \item In the same spirit as \cite{giacoma_toward_2015}
        \item Leverage higher transfer learning capabilities of HiDeNN
    \end{itemize}
    \item Discussion on the orthogonality constraint 
    \item Non zero BCs implementation in NeuROM in \cref{chap:TD} in \cref{BCs_TD}
    \item Different behaviour with non zero BCs
\end{itemize}

\subsection{2 extra parametera}

Works very well (3 modes) for a 2 stiffness structure (left half and right half having different young moduli that can each take any given value in a given parametric range. (how to represent without GiF ?) \cref{chap:TD} in \cref{1D_NeuROM_Bi}
\section{Technical}

\subsection{Code efficiency}

\begin{itemize}
    \item Re-think architecture to vectorise as many functions as possible
    \item Need to use GPUs to get same performaces as FEM solvers \parencite{park_convolution_2023}
\end{itemize}

\subsection{Jean-Zay cluster}

See \cref{chap:Note} in \cref{sec:JZ}.

