 \chapter[The 26$^{\text{th}}$ of April 2024 -2D]{2D \& strategy}

\begin{chapabstract}
	Meeting report
\end{chapabstract}


\minitoc

\section{Presented results}
\paragraph{Potential energy}
In the past two weeks, the primal formulation based on the minimisation of the potential energy has been implemented and shows promising results.
\begin{itemize}
	\item The potential energy minimisation in 2D works well \cref{chap:2DPotentialEnergy}
	\item R-refinement works very well
	\item Multi scale training helps being more efficient
	\item Would get better efficiency (Error/Dofs) by using h-adaptivity
	\begin{itemize}
		\item Local element-wise refinement is not convincing
		\item Multi-element refinement seems promising
		\begin{itemize}
			\item Needs some regularisation 
			\item And/or maximum sub-refinement for a given element
		\end{itemize}
	\end{itemize}
\end{itemize}

\paragraph{Mixed formulation}
The implementation of the mixed formulation as well as our understanding of it improved significantly.
\begin{itemize}
	\item Playing on the weights of the two terms in the loss helps improving the displacement or the stress
	\item Fixes in the technical implementation
	\item \cref{chap:2DPotentialEnergy} shows a better understanding of the quantities at stake
	\begin{itemize}
		\item There might be something to do with an integral formulation of the terms of the loss as opposed to using the RMSE over sampling points.
	\end{itemize}
\end{itemize}

\section{Next TODOs}
It's time to start working more precisely on the papers and their content
\begin{itemize}
	\item Decide on the coarse scale of the papers and their outline
	\begin{itemize}
		\item \emph{A priori} two papers are still required 
		\begin{itemize}
			\item One focusing on reduced-order modelling
			\item The other on the implementation, framework and mixed formulation
			\begin{itemize}
				\item It could be the best place to talk about h-adaptivity
			\end{itemize}
		\end{itemize}
	\end{itemize}
	\item Merge parametric work with 2D results (at least potential energy version)
	\item Try a non-linear problem
\end{itemize}

\section{Ideas for later}
\begin{itemize}
	\item Handling different geometries
	\begin{itemize}
		\item For a given topology we can solve an elastic problem where all the boundary are prescribed Dirichlet conditions forcing the reference geometry to conform to a given new geometry. 
		\item The displacement field (even coarsely converged) would be the mapping from one reference geometry to a new geometry
		\item With the \textsc{pytorch} tool that mapping would still allow using autodifferenciating tools
	\end{itemize}
	\item Use a different optimizer for the nodes coordinates
	\begin{itemize}
		\item Have a different quantity to optimize for the mesh than the total energy
	\end{itemize}
\end{itemize}