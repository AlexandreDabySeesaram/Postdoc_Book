% Variational autoencoder architecture.
% Made with https://github.com/battlesnake/neural.

\documentclass[tikz]{standalone}
\usepackage{stmaryrd} % Where \llbracket and \rrbracket are defined

\usepackage{neuralnetwork}


\newcommand{\xin}[2]{$\Psi_#2$}
\newcommand{\xout}[2]{$\hat \Psi_#2$}
\newcommand{\xhid}[2]{$\mu_#2$}

%%% Colours
\definecolor{BleuLMS}{RGB}{1, 66, 106}
\definecolor{accentcolor}{RGB}{1, 66, 106}
\definecolor{GreenLMS}{RGB}{0,103,127} 
\definecolor{LGreenLMS}{RGB}{67,176,42} 
\definecolor{RougeLMS}{RGB}{206,0,55} 

\begin{document}
	\begin{neuralnetwork}[height=8]
	\tikzstyle{input neuron}=[neuron, fill=GreenLMS, text=white];
	\tikzstyle{output neuron}=[neuron, fill=accentcolor, text=white];
	\tikzstyle{hidden neuron}=[neuron, draw = LGreenLMS, thick, fill=LGreenLMS!25, text=LGreenLMS!60!black];
	
	\inputlayer[count=8, bias=false, title=Input Layer, text=\xin]
	
	\hiddenlayer[count=5, bias=false]
	\linklayers
	
	\hiddenlayer[count=3, bias=false, title=Latent space \\ $\left\{\mu_i\right\}_{i=\llbracket 1, \beta \rrbracket}$,text=\xhid]
	\linklayers
	
	\hiddenlayer[count=5, bias=false]
	\linklayers
	
	\outputlayer[count=8, title=Output Layer, text=\xout]
	\linklayers
	
\end{neuralnetwork}
\end{document}


